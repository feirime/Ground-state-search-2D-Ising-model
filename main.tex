% \documentclass[review]{elsarticle}
\documentclass[utf8, babel, sor, jor, amsmath,amssymb, reprint]{elsarticle} %удалить перед отправкой
\usepackage[T2A]{fontenc} %удалить перед отправкой
\usepackage[utf8x]{inputenc} %удалить перед отправкой
\usepackage[english,russian]{babel} %удалить перед отправкой
\graphicspath{{images/}}

\usepackage{lineno,hyperref}
\usepackage{algorithm}
\usepackage{algorithmic}
\modulolinenumbers[5]

\journal{Journal of \LaTeX\ Templates}

%%%%%%%%%%%%%%%%%%%%%%%
%% Elsevier bibliography styles
%%%%%%%%%%%%%%%%%%%%%%%
%% To change the style, put a % in front of the second line of the current style and
%% remove the % from the second line of the style you would like to use.
%%%%%%%%%%%%%%%%%%%%%%%

%% Numbered
%\bibliographystyle{model1-num-names}

%% Numbered without titles
%\bibliographystyle{model1a-num-names}

%% Harvard
%\bibliographystyle{model2-names.bst}\biboptions{authoryear}

%% Vancouver numbered
%\usepackage{numcompress}\bibliographystyle{model3-num-names}

%% Vancouver name/year
%\usepackage{numcompress}\bibliographystyle{model4-names}\biboptions{authoryear}

%% APA style
%\bibliographystyle{model5-names}\biboptions{authoryear}

%% AMA style
%\usepackage{numcompress}\bibliographystyle{model6-num-names}

%% `Elsevier LaTeX' style
\bibliographystyle{elsarticle-num}
%%%%%%%%%%%%%%%%%%%%%%%


\usepackage{xcolor}
\newcommand{\todo}[1] {\textcolor{red}{#1}} %%for TODO comments
\def\l{\left\langle}
\def\r{\right\rangle}

\usepackage{mathrsfs}
\usepackage{amsmath}
\usepackage{amssymb}%



\begin{document}

\begin{frontmatter}


\title{Ground state search 2D Ising model}

\author[mainaddress, secondaryaddress]{Viacheslav Trukhin\corref{mycorrespondingauthor}}
\ead{trukhin.vo@dvfu.ru}

\author[mainaddress, secondaryaddress]{Konstantin Nefedev\corref{mycorrespondingauthor}}
\ead{nefedev.kv@dvfu.ru}

\address[mainaddress]{Far Eastern Federal University, Vladivostok, Russky Island, 10 Ajax Bay, 690922, the Russian Federation}
\address[secondaryaddress]{Institute of Applied Mathematics, Far Eastern Branch, Russian Academy of Science, Vladivostok, Radio 7, 690041, the Russian Federation}

\begin{abstract}


\end{abstract}


\begin{keyword}
Ising model, GPU and CPU high performance calculations, spin ice, spin glass, statistical thermodynamics.

\end{keyword}


\end{frontmatter}

\linenumbers
\newpage
\tableofcontents

\newpage
\section{Введение}



\section{Решение исчерпывающим перебором}

Модель спинового стекла Эдвардса-Андерсена представляет собой плоскую решетку Изинга:

\begin{equation}
	E = -\sum J_{ij} S_i S_j
	\label{eq:ising_energy}
\end{equation}

, где обменные интегралы $J$ могут принимать значения +1 или -1 создавая таким образом приближение аморфных материалов

Для решения такой цепочки из трёх спинов статистическая сумма
принимает вид:

\begin{equation}
	Z = e^{3\beta - 3\beta h} + 3e^{\beta - h - \beta} + 3e^{\beta h - \beta} + e^{3\beta + 3\beta h}
	\label{eq:stat_3}
\end{equation}

\section{Благодарности}

 

\bibliography{mybibfile}


\end{document}